\documentclass[]{article}
\usepackage{lmodern}
\usepackage{amssymb,amsmath}
\usepackage{ifxetex,ifluatex}
\usepackage{fixltx2e} % provides \textsubscript
\ifnum 0\ifxetex 1\fi\ifluatex 1\fi=0 % if pdftex
  \usepackage[T1]{fontenc}
  \usepackage[utf8]{inputenc}
\else % if luatex or xelatex
  \ifxetex
    \usepackage{mathspec}
  \else
    \usepackage{fontspec}
  \fi
  \defaultfontfeatures{Ligatures=TeX,Scale=MatchLowercase}
\fi
% use upquote if available, for straight quotes in verbatim environments
\IfFileExists{upquote.sty}{\usepackage{upquote}}{}
% use microtype if available
\IfFileExists{microtype.sty}{%
\usepackage{microtype}
\UseMicrotypeSet[protrusion]{basicmath} % disable protrusion for tt fonts
}{}
\usepackage[margin=1in]{geometry}
\usepackage{hyperref}
\hypersetup{unicode=true,
            pdftitle={Donnermeyer\_Homework\_5},
            pdfauthor={Sam Donnermeyer},
            pdfborder={0 0 0},
            breaklinks=true}
\urlstyle{same}  % don't use monospace font for urls
\usepackage{graphicx}
% grffile has become a legacy package: https://ctan.org/pkg/grffile
\IfFileExists{grffile.sty}{%
\usepackage{grffile}
}{}
\makeatletter
\def\maxwidth{\ifdim\Gin@nat@width>\linewidth\linewidth\else\Gin@nat@width\fi}
\def\maxheight{\ifdim\Gin@nat@height>\textheight\textheight\else\Gin@nat@height\fi}
\makeatother
% Scale images if necessary, so that they will not overflow the page
% margins by default, and it is still possible to overwrite the defaults
% using explicit options in \includegraphics[width, height, ...]{}
\setkeys{Gin}{width=\maxwidth,height=\maxheight,keepaspectratio}
\IfFileExists{parskip.sty}{%
\usepackage{parskip}
}{% else
\setlength{\parindent}{0pt}
\setlength{\parskip}{6pt plus 2pt minus 1pt}
}
\setlength{\emergencystretch}{3em}  % prevent overfull lines
\providecommand{\tightlist}{%
  \setlength{\itemsep}{0pt}\setlength{\parskip}{0pt}}
\setcounter{secnumdepth}{0}
% Redefines (sub)paragraphs to behave more like sections
\ifx\paragraph\undefined\else
\let\oldparagraph\paragraph
\renewcommand{\paragraph}[1]{\oldparagraph{#1}\mbox{}}
\fi
\ifx\subparagraph\undefined\else
\let\oldsubparagraph\subparagraph
\renewcommand{\subparagraph}[1]{\oldsubparagraph{#1}\mbox{}}
\fi

%%% Use protect on footnotes to avoid problems with footnotes in titles
\let\rmarkdownfootnote\footnote%
\def\footnote{\protect\rmarkdownfootnote}

%%% Change title format to be more compact
\usepackage{titling}

% Create subtitle command for use in maketitle
\providecommand{\subtitle}[1]{
  \posttitle{
    \begin{center}\large#1\end{center}
    }
}

\setlength{\droptitle}{-2em}

  \title{Donnermeyer\_Homework\_5}
    \pretitle{\vspace{\droptitle}\centering\huge}
  \posttitle{\par}
    \author{Sam Donnermeyer}
    \preauthor{\centering\large\emph}
  \postauthor{\par}
      \predate{\centering\large\emph}
  \postdate{\par}
    \date{10/9/2020}


\begin{document}
\maketitle

\hypertarget{hand-in-homework-4}{%
\subsection{Hand In Homework \#4}\label{hand-in-homework-4}}

\hypertarget{problem-1-sampling-problems-questions-2-3}{%
\subsubsection{Problem \#1: Sampling Problems (Questions 2 \&
3)}\label{problem-1-sampling-problems-questions-2-3}}

\begin{center}\rule{0.5\linewidth}{0.5pt}\end{center}

\hypertarget{section}{%
\paragraph{\texorpdfstring{\textbf{(2)}}{(2)}}\label{section}}

A researcher wishes to estimate the total number of users of a trail
head over the course of 15 weeks. She randomly selects 3 Mondays, 3
Tuesdays, 3 Wednesdays, 3 Thursdays, 3 Fridays, 5 Saturdays and 5
Sundays from the whole 15-week period and stations an observer there on
each selected day. The observer records the number of trail users for
the day.

Population: - All users of a the trail head over the course of the given
15 weeks.

Sampling Units: - Trail users (people).

Sampling Plan: - Multistage - Stratified Random Sampling (To Identify
which days to select) -\textgreater{} Cluster Sampling (Sample All Trail
Users that Day)

\begin{center}\rule{0.5\linewidth}{0.5pt}\end{center}

\hypertarget{section-1}{%
\paragraph{\texorpdfstring{\textbf{(3)}}{(3)}}\label{section-1}}

In the previous problem, the researcher would also like to estimate the
proportion of trail users who are from out of state. The observer also
records where each trail user is from.

Population: - All users of the trail head over the course of the given
15 weeks who are from out of state.

Sampling Units: - Trail users (people).

Sampling Plan: - Multistage - Stratified Random Sampling (To Identify
which days to select) -\textgreater{} Cluster Sampling (Sample All Trail
Users that Day)

\hypertarget{problem-2-part-iii-review-exercises-pp.-373-378-r.3.10-r.3.12-r.3.16-r.3.28-r.3.32}{%
\subsubsection{Problem \#2: Part III Review Exercises (pp.~373-378):
R.3.10, R.3.12, R.3.16, R.3.28,
R.3.32}\label{problem-2-part-iii-review-exercises-pp.-373-378-r.3.10-r.3.12-r.3.16-r.3.28-r.3.32}}

Analyze the design of each research example reported. Is it a sample
survey, an observational study, or an experiment? If a sample, what are
the population, the parameter of interest, and the sampling procedure?
If an observational study, was it retrospective or prospective? If an
experiment, describe the factors, treatments, randomization, response
variable, and any blocking, matching, or blinding that may be present.
In each, what kind of conclusions can be reached?

\hypertarget{r.3.10}{%
\paragraph{R.3.10}\label{r.3.10}}

The study detailed in the problem is a retrospective observational study
because the men who had prostate cancer were identified through an
existing cancer registry. There seems to be strong evidence that men who
have vasectomies is connected to developing prostate cancer.

\hypertarget{r.3.12}{%
\paragraph{R.3.12}\label{r.3.12}}

This is an experiment with the factors, glaze and the temperatures the
pottery was baked at. The different treatments available are the four
types of glazes and the three different temperature settings in the
kiln. There was no randomization included in this experiment. The
response variable is the `appearance of age' in the newly made pottery.
There was no blocking, matching, or blinding present in the experiment.
Based on the results of the experiment, the potter can determine which
combination of factors and treatments creates pottery that appears to be
aged.

\hypertarget{r.3.16}{%
\paragraph{R.3.16}\label{r.3.16}}

The scenario detailed is a sample survey, where the population is all
soda bottles produced by the soft-drink manufacturer and the parameter
of interest is strength of the caps on the newly bottled soda.

\hypertarget{r.3.28}{%
\paragraph{R.3.28}\label{r.3.28}}

\begin{enumerate}
\def\labelenumi{\alph{enumi})}
\item
  The factors are the cooking times (levels: 10 \& 15 minutes),
  different cooking temperatures (levels: 325, 375, and 425 degrees F),
  and the amount of chocolate chips in each cookie (levels: 5 or 10).
\item
  Blinding in the context of this experiment means the evaluators (the
  six friends) are unaware of the different levels in each of the
  factors in order to prevent any bias from entering the evaluation. It
  is not double blinded because Mary Beth, NIgel, and Molly know what
  types of cookies they are servering and have made in order to evaluate
  them properly after the six friends give their feedback.
\item
  Each judge will taste 12 different cookies in order to be exposed to
  all levels of each factor.
\item
  The response variable will be ordinal, to show the order in which the
  judges deem the cookies to be in, to rank the cookies from from best
  to worse.
\item
  The challenges of the experiment is the resources to make 72 cookies
  (12 for each judge) and carry out the processes correctly and
  consistently, essentially the reproducability of each type of cookie.
  Secondly, the judge's might not be able to differentiate between the
  cookies enough to give any meaningful information regarding the best
  cookie combination.
\end{enumerate}

\hypertarget{r.3.32}{%
\paragraph{R.3.32}\label{r.3.32}}

\begin{enumerate}
\def\labelenumi{\alph{enumi})}
\item
  This is an experiment which has a block, which is the sex of the
  volunteer.
\item
  No, the experiment is to resolve the effectiveness of the treatment
  among those who suffer from reflux disease, and since all volunteers
  are suffering from reflux disease, the experiment can determine if the
  treatment aids the reflux disease.
\item
  The use of a placebo would make the volunteers blind to the treatment
  that they are receiving. This would allow for the results from the
  antacid to be vetted.
\end{enumerate}


\end{document}
